\documentclass[conference]{IEEEtran}
\IEEEoverridecommandlockouts
% The preceding line is only needed to identify funding in the first footnote. If that is unneeded, please comment it out.
\usepackage{cite}
\usepackage{amsmath,amssymb,amsfonts}
\usepackage{algorithmic}
\usepackage{graphicx}
\usepackage{textcomp}
\usepackage{xcolor}
\usepackage{subfig}
\usepackage{svg}
\usepackage{multirow}

\usepackage{multicol}


\usepackage{listings}

\usepackage{caption}

\usepackage{listings}

\lstset{
  basicstyle=\ttfamily,
}
\renewcommand*{\lstlistingname}{Code}

\definecolor{RoyalBlue}{RGB}{65,105,225}
\definecolor{Blue}{RGB}{0,0,225}
\definecolor{YellowGreen}{RGB}{154,205,50}
\definecolor{ForestGreen}{RGB}{34,139,34}


\lstset{ 
  linewidth=8.5cm,
  language=R,                     % the language of the code
  basicstyle=\scriptsize\ttfamily, % the size of the fonts that are used for the code
  numbers=left,                   % where to put the line-numbers
  numberstyle=\scriptsize\color{Blue},  % the style that is used for the line-numbers
  stepnumber=1,                   % the step between two line-numbers. If it is 1, each line
                                  % will be numbered
  numbersep=5pt,                  % how far the line-numbers are from the code
  backgroundcolor=\color{white},  % choose the background color. You must add \usepackage{color}
  showspaces=false,               % show spaces adding particular underscores
  showstringspaces=false,         % underline spaces within strings
  showtabs=false,                 % show tabs within strings adding particular underscores
  frame=single,                   % adds a frame around the code
  rulecolor=\color{black},        % if not set, the frame-color may be changed on line-breaks within not-black text (e.g. commens (green here))
  tabsize=2,                      % sets default tabsize to 2 spaces
  captionpos=b,                   % sets the caption-position to bottom
  breaklines=true,                % sets automatic line breaking
  breakatwhitespace=false,        % sets if automatic breaks should only happen at whitespace
  keywordstyle=\color{RoyalBlue},      % keyword style
  commentstyle=\color{YellowGreen},   % comment style
  stringstyle=\color{ForestGreen}      % string literal style
} 

\usepackage{amssymb}% 
\usepackage{pifont}% 
\newcommand{\cmark}{\ding{51}}%
\newcommand{\xmark}{\ding{55}}%


\def\BibTeX{{\rm B\kern-.05em{\sc i\kern-.025em b}\kern-.08em
    T\kern-.1667em\lower.7ex\hbox{E}\kern-.125emX}}
\begin{document}

\title{Windows vs Linux approach for image rendering}

\author{
\IEEEauthorblockN{Allan Sánchez Masís}
\IEEEauthorblockA{\textit{Computer Science School} \\
\textit{Instituto Tecnológico de Costa Rica}\\
allansanchezmasis@gmail.com}

}

\maketitle

\begin{abstract}
This paper shows a comparison through image rendering time, of different Operating System implementations; Compiled to run natively on Windows, in a Ubuntu virtual machine and Debian virtual machine with Windows as a host, and compiled for Ubuntu, running on Windows Linux Subsystem. The comparison is to evaluate the performance of Linux, Windows, virtual machine and Windows Linux subsystem, via statistical analysis. 

\end{abstract}

\begin{IEEEkeywords}
Statistical analysis, Linux, Windows, image rendering
\end{IEEEkeywords}

\section{Introduction}
Understanding the Operating System (OS) is crucial in computer sciences because the OS is what makes the
hardware work with the software like an
interface that allows the executions of process \cite{dhamija2012demographics}.
\section{Conclusions}
The designed 


\bibliographystyle{IEEEtran}
\bibliography{bibliography.bib}
%https://www.quora.com/How-can-I-test-data-for-a-uniform-distribution
%http://rstudio-pubs-static.s3.amazonaws.com/433558_30d5068dd9fe45d58243c018c7582fc0.html
%https://stackoverflow.com/questions/44912747/plot-histogram-for-discrete-data

\end{document}

