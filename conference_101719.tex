\documentclass[conference]{IEEEtran}
\IEEEoverridecommandlockouts
% The preceding line is only needed to identify funding in the first footnote. If that is unneeded, please comment it out.
\usepackage{cite}
\usepackage{amsmath,amssymb,amsfonts}
\usepackage{algorithmic}
\usepackage{graphicx}
\usepackage{textcomp}
\usepackage{xcolor}
\usepackage{subfig}
\usepackage{svg}
\usepackage{multirow}

\usepackage{multicol}


\usepackage{listings}

\usepackage{caption}

\usepackage{listings}

\usepackage{lipsum}
\usepackage{graphicx}
\ifCLASSOPTIONcompsoc
    \usepackage[caption=false, font=normalsize, labelfont=sf, textfont=sf]{subfig}
\else
\usepackage[caption=false, font=footnotesize]{subfig}
\fi


\lstset{
  basicstyle=\ttfamily,
}
\renewcommand*{\lstlistingname}{Code}

\definecolor{RoyalBlue}{RGB}{65,105,225}
\definecolor{Blue}{RGB}{0,0,225}
\definecolor{YellowGreen}{RGB}{154,205,50}
\definecolor{ForestGreen}{RGB}{34,139,34}


\lstset{ 
  linewidth=8.5cm,
  language=R,                     % the language of the code
  basicstyle=\scriptsize\ttfamily, % the size of the fonts that are used for the code
  numbers=left,                   % where to put the line-numbers
  numberstyle=\scriptsize\color{Blue},  % the style that is used for the line-numbers
  stepnumber=1,                   % the step between two line-numbers. If it is 1, each line
                                  % will be numbered
  numbersep=5pt,                  % how far the line-numbers are from the code
  backgroundcolor=\color{white},  % choose the background color. You must add \usepackage{color}
  showspaces=false,               % show spaces adding particular underscores
  showstringspaces=false,         % underline spaces within strings
  showtabs=false,                 % show tabs within strings adding particular underscores
  frame=single,                   % adds a frame around the code
  rulecolor=\color{black},        % if not set, the frame-color may be changed on line-breaks within not-black text (e.g. commens (green here))
  tabsize=2,                      % sets default tabsize to 2 spaces
  captionpos=b,                   % sets the caption-position to bottom
  breaklines=true,                % sets automatic line breaking
  breakatwhitespace=false,        % sets if automatic breaks should only happen at whitespace
  keywordstyle=\color{RoyalBlue},      % keyword style
  commentstyle=\color{YellowGreen},   % comment style
  stringstyle=\color{ForestGreen}      % string literal style
} 

\usepackage{amssymb}% 
\usepackage{pifont}% 
\newcommand{\cmark}{\ding{51}}%
\newcommand{\xmark}{\ding{55}}%


\def\BibTeX{{\rm B\kern-.05em{\sc i\kern-.025em b}\kern-.08em
    T\kern-.1667em\lower.7ex\hbox{E}\kern-.125emX}}
\begin{document}

\title{Performance of Windows vs Linux approach for image rendering}

\author{
\IEEEauthorblockN{Allan Sánchez Masís}
\IEEEauthorblockA{\textit{Computer Science School} \\
\textit{Instituto Tecnológico de Costa Rica}\\
allansanchezmasis@gmail.com}

}

\maketitle

\begin{abstract}
This paper shows a comparison through image rendering time, of different Operating System implementations; Compiled to run natively on Windows, in a Ubuntu virtual machine and Debian virtual machine with Windows as a host, and compiled for Ubuntu, running on Windows Linux Subsystem. The comparison is to evaluate the performance of Linux, Windows, virtual machine and Windows Linux subsystem, via statistical analysis. 

\end{abstract}

\begin{IEEEkeywords}
Linux, Windows, and Virtual Machines for image rendering
\end{IEEEkeywords}

\section{Introduction}
Understanding the Operating System (OS) is crucial in computer sciences because the OS is what makes the
hardware work with the software like an
interface that allows the executions of processes \cite{dhamija2012demographics}. Thus, it is necessary to understand which OS gives the best performance and facilities according with the different user tasks, it is critical to focus more the researches not only in their qualities, also with numbers and metrics to quantify their performance. \par
There are different well-known features which OS like Windows and Linux offer for the different spectrum of users and their needs. There are very well-known features of this OS, for example, Linux is free to obtain, meanwhile, Microsoft products are available for a considerable fare \cite{dhamija2012demographics}, and the licenses has a price for every machine where are installed  \cite{duran2006analisis}, on the other hand, Windows is more popular than Windows \cite{duran2006analisis}, thus, typical users prefers Windows because it is what they probably will need for their office jobs, studies, games, and so on, therefore, people is not much familiar with Linux as Windows and require more expertise and learning curve \cite{dhamija2012demographics}.\par
Considering technical aspects, vast of Linux programs are Open Source, thus, users are able to modify the feature of different software packages according to their necessities, nevertheless, Linux can only run binaries that have been  created for it, whereas Windows has a lot of of programs, significantly more than any other OS \cite{dhamija2012demographics}.\par
This paper presents a comparison of different OS, the evaluation is done measuring the rendering time of PBR  program specifically, the third version (PBRT-v3) \cite{PBRT}, the scenarios are five: Compiled to run natively on Windows, also natively on Ubuntu, in an Ubuntu virtual machine (U-VM) and a Debian virtual machine (D-VM) using Windows as the host, as well as compiled for Ubuntu operating on the Windows Linux Subsystem (U-WLS). This scenarios under test are studied with 2 factors, the accelerator and the resolution used in PBRT-V3. To make the analysis it is used statisticall analysis based onAnova

\begin{figure} 
    \centering
  \subfloat[a\label{1a}]{%
       \includegraphics[width=0.45\linewidth]{example-image}}
    \hfill
  \subfloat[b\label{1b}]{%
        \includegraphics[width=0.45\linewidth]{example-image}}
    \\
  \subfloat[c\label{1c}]{%
        \includegraphics[width=0.45\linewidth]{example-image}}
    \hfill
  \subfloat[d\label{1d}]{%
        \includegraphics[width=0.45\linewidth]{example-image}}
  \caption{(a), (b) Some examples from CIFAR-10 \cite{4}. The objects in     
        single-label images are usually roughly aligned.(c),(d) However, the 
        assumption of object alignment is not valid for multi-label
        images. Also note the partial visibility and occlusion
        between objects in the multi-label images.}
  \label{fig1} 
\end{figure}




\section{Conclusions}
The designed 


\bibliographystyle{IEEEtran}
\bibliography{bibliography.bib}
%https://www.quora.com/How-can-I-test-data-for-a-uniform-distribution
%http://rstudio-pubs-static.s3.amazonaws.com/433558_30d5068dd9fe45d58243c018c7582fc0.html
%https://stackoverflow.com/questions/44912747/plot-histogram-for-discrete-data

\end{document}

